% !TEX encoding = UTF-8 Unicode

%!TEX TS-program = xelatex
%!TEX encoding = UTF-8 Unicode

\documentclass[oneside,12pt]{book}
\usepackage[left=2cm,top=1cm,right=3cm,nofoot]{geometry}                % See geometry.pdf to learn the layout options. There are lots.
\geometry{a4paper}                   % ... or a4paper or a5paper or ... 
\usepackage{tabularx}

%\usepackage{fontspec,xltxtra,xunicode}
%\defaultfontfeatures{Mapping=tex-text}
\usepackage[french]{babel}
\usepackage{listings}
\usepackage{calc}
\usepackage{graphicx}
\usepackage[linktocpage]{hyperref}



\title{DDays: Scénario}
\author{Renaud "ObiWan" Guezennec}
\date{}


\begin{document}

\maketitle \clearpage
\tableofcontents \clearpage

\begin{flushleft}
    \chapter{Introduction}
        \section{Introduction}
        \section{Les personnages}
        
      


\clearpage




%% http://www.livejournal.com/tools/memories.bml?user=innocent_man&keyword=Forsaken&filter=all





\chapter{Scénario 1 : La cité des anges!!}

Mots-clés: Init du jdr, init de Dday, introduction d’une campagne road trip.

Pj: 5 agents du majestic.

Les pjs sont parachutés en Pologne. 
Leur mission: identifier les causes des troubles ressentis par des agents infiltrés en mission dans la région. 
Nous supposons qu’à des distances raisonnables des jumpers expérimentés ont détecté des jumps sauvages. 
Ils n’ont pas le temps d’enquêter, on envoie donc une équipe de bleus régler ce problème. 
Certainement un jumper sauvage. 
\section{Les PNJ importants ou pas}
\begin{itemize}
\item Halina Poniatowski : fille de Traby, parle un peu anglais et polonais
\item Traby Poniatowski : ancien soldat, parle anglais et polonais
\end{itemize}


\section{On se retrouve en bas!}
Ils sautent en parachuttent, à l’aube. 
Les joueurs atterrissent en parachute. La zone d'atterrissage est marqué par des projecteurs pour aider les pilotes. La famille 
Poniatowski aident souvent les services secrets britanique pour 
 Ils doivent repérer la zone. 
Halina Poniatowski les attends cachée dans un fossé. 


\section{On recherche des jumpers!}
Laisser les joueurs trouver des éléments d’enquêtes et des pistes. 
Ils peuvent apprendre des informations sur le passé de la vallée. 
Les russes qui occupent la zone. Ils interdisent la route vers la vallée voisine. 
C’est la scène idéale pour introduire des personnes qui vont sembler important aux joueurs: le résistant, la fille du résistant....

\section{On s’est fait eu, chef!}
L’armée rouge débarque dans la ferme (officiellement ils ne sont pas des ennemis). 
Ils cherchent les “anglais”. 
Le groupe de Pj est caché dans le sous-sol de la ferme, le résistant est secoué par les soldats. 
La fille stresse grave devant les soldats, elle implore aux russes de partir. 
Au moment ou les soldats trouvent la trappe qui mène aux pj, elle jump entraînant avec elle quelques russes dont l’officier.


\section{Elle jumpe, elle meurt!}
L’officier tuera la fille dès le saut fini. Les pjs la suivent dans son saut et peuvent agir pour la sauver. 
S’ils sautent après coup, ils la retrouvent morte. Visiblement l’officier a pété un cable devant la scène.
On est encerclé chef !
Les Pjs arrivent et sont clairement en infériorité numérique. Encerclé par des soldats, l’officier leur dit dans un anglais approximatif, qu’il va les mener vers le centre de recherche.
Les voitures ?
L’officier retournera avec ses hommes à sa base qui sera pour le coup, totalement vide. 
Il laisse des hommes pour “éduquer” le village, en effet voler les véhicules de l’armée rouge, c’est pas bien!  
Ils “réquisitionnent” des véhicules pour rentrer. 
Le doute s’installe chez certains soldats.

\section{On rentre du boulot!}
Pendant le trajet, il ne serait pas mal de tisser des liens entre les quelques soldats russes qui trouve qu’il y a des trucs qui clochent.  
Un barrage routier absent, une maison en parfait état, etc...
Les Pj peuvent jouer certaines cartes pour négocier avec les militaires dans la confusion.
Allo, il y a quelqu’un ?
Un bunker dans la montagne, une ou deux vallées plus loin.
La porte d’entrée du bunker est bouchée par un éboulis.
L’officier pète un plomb, tire avec son arme sur les pierres puis ordonnent à tout le monde de creuser (même les Pj). L’officier accusera les british d’avoir fait le coup et menacera les PJ.
Nous avons le savoir, obéissez nous !!
C’est le moment, où les pj peuvent vraiment faire quelques choses: retourner des soldats contre l’officier, négocier avec l’officier et lui faire entendre raison sur le jump.  Tuer le chef..
Chef, il y a des trucs louches !
L’agitation dans les lieux apporte deux choses. Un vieil homme charmant arrive et demande ce qu’il se passe. 
Il peut expliquer (avec un fort accent allemand), que les SS avait établie un laboratoire ici, quand les camps on était découvert par les alliés, les villageois du coin, on fait s’effondrer l’entrée du bunker. 
Le vieux se présente sous le nom de Józef Młynarczyk. Il passe pour être un éleveur de vaches. (Il a vraiment des vaches).  
Pour le MJ:
(En réalité, le vieux est l’ancien patron du laboratoir (Klauss Zilberg), il a fait explosé l’entrée pour dissimuler les traces, sachant que l’armée russe/polonaise a mi un certain temps avant de le trouver. 
C’est une petite structure).
Le vieux cherche a savoir ce qui pousse tant de gens à s’intéresser au camp.

\section{Chef, il y a des trucs louches (bis) !}
Les Pj peuvent entendre des cris qui viennent de l’intérieur de la grotte.
Il s’agit de prisonniers qui sont tous habillés d’uniformes allemands (pas à leur taille). Il y a une forte proportion de femmes et d’enfants.
\section{Méchant le monsieur de l’autre côté!}
Ils expliquent que c’est le paradis ici, ils ont chaud, à manger pour des mois (plein de rations de survie). Qu’ils ne veulent pas retourner là-bas.
si un enfant s’approche un peu trop près du vieux (ou vis versa), il jumpe. Le vieux lui ne le peut pas ( à moins d’être obliger).

\section{Pour le Mj}

\subsection{Le bunker côté D-world}
C’est un laboratoire qui est tenu par les russes mais le chef technique du labo est un officier allemand (Klauss Zilberg). Il essaie de comprendre le jump. Il est persuadé d’avoir établie la preuve mathématique de l’existance du jump. Les Pj auront sûrement à nettoyer la zone.
Laisser les être inventif pour imaginer comment ils vont faire pour nettoyer cela. Vont-ils laisser les prisonniers mourir ? Il y a un certains nombres de prisonniers datant de la guerre et les russes en fournisse un peu aussi.
Le professeur n’a visiblement pas compris tous les points techniques du jump. De plus, personne (ou peu) est revenu après avoir disparue de ses yeux. Il n’a pas compris que ne peuvent sauter que les personnes qui n’ont pas de double de l’autre coté.

\subsection{Le bunker côté O-world}
Le professeur (qui vit à proximité sous un faux nom) sera très intéressé par fouiller dans le bâtiment.
Il y croisera des enfants qu’il a lui même tué (et risque de péter un cable sévère). Ses victimes auront très peur de lui (même s’il est difficile à reconnaître dans ses vêtements de paysans).  

\subsection{Fin}
Au choix des PJ, pour la récupération, c’est à vous de voir. Cela peut être l’occasion d’un road trip avec des victimes (très long et difficile mais cela pourrait donner un bon moyen, au joueur d’utiliser leur Tau). Les joueurs peuvent décider d’exterminer tout le monde (ou d’en garder un ou deux les plus vieux et utile). Les enfants pourraient faire des caprices avant de partir, histoire d’être bien chiant. Ils veulent la photo de leur maman, les doudoux....



















\chapter{Où est mon fils?}


Lieu: From Liverpool to Londres. \\
Tags: Init du jdr, init de Dday \\

Pj: 5 agents du majestic. \\


\section{Briéfing à Londres}
Le rapatriement des enfants anglais est en cours, des bateaux venus de tout le Commonwealth (Australie, Nouvelle-Zéalande, Irlande du nord...) arrivent dans tous les ports du royaume. Le désastre est la raison de ce retour si tardif. 
Les bateaux débarquent à Liverpool, les enfants sont ensuite acheminés en train vers Londres et autres grandes villes. 
Le pays est dans un désordre assez important: Des militaires américains partout.
Les Joueurs doivent aller enquêter sur la disparition d’un groupe d’enfants. D’après les registres, ils sont biens arrivés à Liverpool, ils ont bien pris le train mais ne sont jamais arrivés à Londres.
Le groupe d’enfants perdus est la section 14 de Perth (P14), le bateau qui les a ramené s’appelle le Serenity.


\section{Chef, on fait quoi ?}
Le but, ici, est de laisser les joueurs choisir un peu comment, ils vont s’y prendre pour trouver des indices. Globalement, soit ils vont directement à Liverpool, soit ils remontent dans l'autre sens. 

\section{Liverpool !}
Si les joueurs veulent se rendre à Liverpool, les services peuvent leur laisser un avion  pour y aller plus rapidement. Ils pourront/doivent découvrir plusieurs choses sur place.

\section{L'armée}
L'armée organise le rapatriement. elle ne va pas vraiment avoir le temps de s’occuper d’eux. Ils peuvent demander un accès aux données des registres militaire et voir le numéro de départ du train.

\subsection{Les Indices}
Le registre est un élément qui apportera des indices sur la suite. 
Il contient les données  suivantes:\\
Date arrivée: section : nom bateau: nombre enfants : destination train : date/heure \\
départ train : numéro train : signataire (le militaire qui a rempli le papier)\\

\subsection{Le signataire}
Il dira que tout s'est passé normalement, qu'il a signé le document avec la 

\subsection{L'armée}
Ils peuvent soit aller discuter avec le signataire, qui confirmera avoir fait partir les enfants. 
Le mec est louche mais il trafique du matériel; pas des enfants. 
Les joueurs pourrait avoir des soupçons sur lui.
En contactant les autorités ferroviaires, les joueurs peuvent apprendre le parcours du train.
Il y avait plusieurs escales, changements.
Le but est de faire naître le doute que l’administration a peut-être juste perdu ces enfants.

….





























La voiture, quelle voiture ?  
civils qui découvrent le jump






Le premier jumper ?



\end{flushleft}
\end{document}
