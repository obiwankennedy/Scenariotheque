\documentclass[oneside,12pt]{book}
\usepackage[left=5mm,top=5mm,right=5mm,nofoot]{geometry}                % See geometry.pdf to learn the layout options. There are lots.
\geometry{a4paper}                   % ... or a4paper or a5paper or ...
\usepackage{tabularx}
%\usepackage[latin1]{inputenc}
%fontspec,\usepackage{xltxtra,xunicode}
%\defaultfontfeatures{Mapping=tex-text}
%\usepackage[french]{babel}
\usepackage[utf8]{inputenc}
\usepackage[francais]{babel}
%\usepackage[cyr]{aeguill}
\usepackage{listings}
\usepackage{color}
\usepackage{graphicx}
\usepackage[linktocpage]{hyperref}

\setlength{\parindent}{20pt}

\newcommand\don[6]{
\textbf{#1} \\
(#6) - #2
\begin{itemize}
\item{ \textbf{jet}: #3}
\item{ \textbf{Cout}: #4}
\item{ \textbf{Page}: #5}
\end{itemize}
\vspace{0.5cm}
}
\newcommand\rituel[6]{
{\footnotesize \textbf{#1} (Niveau: #6) \\
#2
\begin{itemize}
\item{ \textbf{jet}: Harmonie}
\item{ \textbf{Succès}:  #3}
\item{  \textbf{Description}: #4}
\item{ \textbf{Page}: #5}
\end{itemize}}
\vspace{0.5cm}
}
\newcommand\roll[1]{
( Jet: \textbf{#1})
}
%%%%%%%%%%%%%%%%%%%%%%%%
% Definition des variables
%%%%%%%%%%%%%%%%%%%%%%%%
\newcommand{\Glen}{\textbf{GrayWolf} }
\newcommand{\Mathew}{\textbf{Sad Song} }
\newcommand{\Laura}{\textbf{blood queen} }
\newcommand{\Andy}{\textbf{Faceless} }
\newcommand{\Kelly}{\textbf{911} }


\title{Créatures Universelles}
\author{Renaud "ObiWan Kennedy" Guezennec}
\date{}

%\let\origdescription\description
%\renewenvironment{description}{
%  \setlength{\leftmargini}{0em}
%  \origdescription
%  \setlength{\itemindent}{1em}
%}


\begin{document}

\maketitle \clearpage
\tableofcontents \clearpage

\begin{flushleft}
    \chapter{Introduction}
    \section{Le scénario en quelques mots}
Les personnages arrivent en ville pour essayer de raisonner le gouverneur et essayer de comprendre le silence radio de leurs contacts.\\
Le gouverneur a autorisé la chasse aux loups sur l’ensemble du territoire de l’État, y compris les parcs naturels, donc le territoire de la meute des joueurs. \\
Leur enquête sur le silence radio des autres meutes va les mener à poursuivre la piste de disparitions. \\
Ces disparitions sont l’œuvre d’un mage ayant le projet d’unir les loup-garous, les mages et les vampires. \\
Les unir physiquement, dans un corps unique.
    \section{Lire introduction sur l'univers de loup-garou}
       Les loups-garous sont mi-humains mi-esprits loups. 
       Il y a très longtemps, à une époque que certains nomment la Pangée, 
       Père Loup faisait régner l'ordre et l’équilibre entre le monde physique et le monde des esprits.\\ 
       Conquise par son courage, Mère Lune tomba amoureuse de lui et prit forme humaine pour le séduire. 
       Elle donna naissance à neuf enfants (les Premiers-Nés). 
       De leur mère ils reçurent la faculté de changer de forme et de leur père la force et l'instinct de chasseur.\\ 
       Le passage des siècles vit la naissance de plusieurs générations de loups-garous, 
       dans lesquelles le pouvoir de Père-Loup se dilua. 
       Une partie de sa meute réalisa qu'il était trop faible pour les guider et lui proposa son aide, mais il la refusa. Alors ses enfants assassinèrent Père-Loup car il refusait de partager son fardeau avec le fruit de sa chair. Lorsqu’il reçut le coup mortel, son cri fut si puissant qu’il déchira le monde. Depuis lors, Hisile, le monde des esprits, s’est éloigné du monde physique.\\ Les esprits en veulent aux loups-garous (urathas) car ils ont anéanti leur paradis et sont parvenus à tuer Père-Loup alors qu’eux n’y étaient jamais arrivés. Mère-lune maudit ses propres enfants pour les punir de leur crime et les rendit vulnérables à l'argent, le métal le plus précieux à ses yeux.\\ 
       Au fil des siècles, elle allégea la malédiction de ceux de ses enfants qui reprirent la tâche de son bien-aimé. 
       Ces loups-garous là sont les gardiens de l’équilibre entre le monde des esprits et le monde physique.  \\
   \subsection{L'hisile}    
Une nuit éternelle règne sur le monde des esprits. 
À l’exception des humains, dont les émotions l’alimentent, chaque élément du monde physique y trouve son reflet. 
Les esprits peuvent être des concepts comme la violence, la jalousie ou l'appétit, ou bien être le reflet d’objets ou d’animaux, comme une porte ou un renard.\\ Les loups-garous sont les seuls êtres autorisés à passer d’un monde à l’autre.

\subsection{Les atouts d'un loup-garou}
Quand je fais jouer ce scénario à des débutants, je précise les avantages d'etre loup-garou (voir les règles du jeu).
\begin{itemize}
\item Les 5 formes : humaine, presque humaine, Loup-garou, Presque loup, loup.
\item Locus : portail pour passer du monde physique au monde des esprits, c'est le coeur d'un territoire loup-garou et également la source de leur énergie magique (l'essence). 
\item Régénération : Le corps des loups-garous est plus résistant qu'un humain, les blessures se soignent bien plus vite.
\item Tribu : un groupe de loups-garous qui partage une même philosophie de vie. Il en existe cinq.
\end{itemize}

\subsection{Les ennemis}
\begin{itemize}
\item Les esprits : leur but est de grandir le plus possible, pour cela ils sont prets à tout. 
\item Les purs : Loups-garous qui refusèrent de prendre part à la mort de Père-Loup et qui refusent de maintenir l'équilibre.
\item Humains : ils ignorent existence des loups-garous. S’ils la découvrent, ils auront des intentions hostiles.
\end{itemize}


\subsection{Serment de la Lune}
Un loup-garou déchu a preté ce serment. C'est un ensemble de règles (libre d'interprétation) à respecter pour vos joueurs.
\begin{itemize}
\item Les loups doivent chasser
\item Le peuple ne tue pas le peuple
\item Le faible honore le fort, Le fort respecte le faible.
\item Respecte ta proie
\item Le peuple doit vivre parmi les humains
\item Ne mange pas de chair humaine ou de loup
\item Le troupeau ne doit pas savoir
\end{itemize}

\section{Les meutes}
\subsection{La Meute PJ}
Glen Taylor : GrayWolf : Alpha - Elodoth - chasseur des ténèbres\\
Mathew LittleHawk : Sad Song : Cahalithe - Os de l’Ombre \\
Laura Bennett : Blood Queen : Rahu - Seigneur des Tempêtes\\
Andy Zeiner : Faceless : Irraka - Maître du fer\\
Kelly Clark : 9/11 : Ithaeur - Griffe de Sang\\

\subsection{Les autres meutes}
\begin{itemize}
\item Whispers \& Ashes (Chuchotements et cendres) : meute de Chasseur des ténèbres.
\item Weathermen (Faiseurs de temps) : meute de Seigneur des tempêtes.
\item Rooftop riders: (Coureurs sur toit). : meute de Maître du fer.
\end{itemize}


\clearpage
\section{Elodoth}
\begin{description}
\item[Nom:]{Glen Taylor}
\item[Origine:]{Canada}
\item[Auspice:]{Elodoth}
%\item[Os:]{Protecteur de la nature}
%\item[Sang:]{Père}
%\item[Attache Spirituelle:]{Ta meute}
%\item[Attache Matérielle:]{Les animaux domestique/du parc}
\item[Tribu:]{Chasseur des ténèbres}
\item[Profession:]{Gérant d'un sanctuaire pour animaux}
\item[Nom de guerre:]{GrayWolf}
\item[Age:]{38 ans}
\item[Histoire:]{
Depuis les morts de ta femme et ta fille, le 22 avril 2000 dans un tragique accident routier, tu t’es effondré.\\
-Tu as cassé la gueule à ton patron (parce qu’il n’avait pas respecté les normes de sécurité dans le zoo où tu travaillais). \\
-Tu t’es faché avec tes amis qui essayaient de te changer les idées.\\
Au chômage, sans argent et n’ayant plus le désir de vivre, tu es devenu SDF, te laissant mourir.
Le peu d’argent que tu avais aller directement dans l’alcool.\\ 
Très vite, tu as commencé à récupérer les chats et chiens errants. Tu les nourrissais comme tu pouvais. Ça avait l’avantage de toucher un peu les passants, ils se montraient généreux avec toi.
Un soir, une bande de jeunes s’est attaquée à tes animaux. Tu voulais les aider mais tu étais complètement rond par l’alcool. C'est à ce moment qu'une grande colère s’est réveillée en toi, tu baignais dans ton urine, ton sang, ta crasse et pourtant la lune était… si belle sous sa forme de moitié. Ta vue s’éclaircit par l'énergie de la colère puis le black-out total. Le lendemain, tu étais couvert de sang au milieu des cadavres des petits cons. \\
Une meute t’a récupéré pour t’apprendre ta nouvelle vie. Tu es devenu membre de la meute. Puis, elle s’est faite décimer car ils ont refusé de suivre ton conseil face à un danger. Tu as survécu de justesse. La mort ne te veut vraiment pas, on dirait!\\
Tu as voulu t’éloigner de la ville et tu as commencé à vivre en ermite proche d'une réserve indienne. \\
Tu avais négocier avec les meutes voisines pour vivre tranquillement, tout seul. Un jour un jeune loup-garou venu de la réserve vint te voir, il avait besoin de toi pour se venger et «détruire» un esprit. Sa tante lui avait dit avant de mourir que tu avais la solution pour battre cet esprit. En réalité, tu étais la solution, l’esprit en question ne pouvait affecter que les «amérindiens». L'esprit n'a rien pu faire contre toi. Le gamin a pris cela pour de la bravoure et il t’a accepté comme alpha. Cela ne t’intéressait pas cependant tu étais le seul candidat possible dans toutes la région. Les meutes des environs avaient été détruites par l’esprit.
Vous avez petit à petit recréer une vraie meute. 
Depuis plusieurs mois, tu tiens dans le parc naturel du «Glacier» un refuge pour animaux. \\
La chasse aux loups a été autorisée par le gouverneur de l’État. Ce fut un véritable massacre. Tu as sauvé ce que tu as pu mais c'est jamais assez. Ces troubles salissent votre territoire, il est temps de prendre les devants. Tu as donc décidé d'aller voir se gouverneur et lui dire d’arrêter ses conneries. La meute n’est pas super chaude pour aller en ville mais bon ils savent que c’est nécessaire. \\

Avis:\\
\underline{\Mathew} : Un jeune idéaliste, sentiment mitigé entre paternalisme et ennuie.\\
\underline{\Laura} : tu aimes bien son côté looseuse. Elle est de bon conseil, elle connaît la difficulté d'être un alpha.  \\
\underline{\Andy} : Type très secret. Il a du mal à partager avec la meute. Il ne semble pas faire confiance aux autres. \\
\underline{\Kelly} : Une femme forte, elle fera sûrement une très bonne alpha si elle surpasse sa mélancolie. \\
}
\end{description}
\clearpage
\textbf{\large Dons de Lune} \\
\vspace{0.5cm}
\don{Aura de trêves}{Projette une aura de calme, efficace avant que le combat commence. Les personnes doivent dépenser un point de volonté pour commettre un acte violent}{Manipulation + Persuasion + Honneur}{1 Point d’Essence}{113}{Instantanée}
\don{Sentir la corruption}{Identifier la nature d'un individu: vampire ou autre...}{Astuce + Occulte + Pureté}{Aucun}{136}{Instantanée}
\don{Vision nocturne}{Voir dans la nuit en forme humaine}{Intelligence + Survie + Honneur}{aucun}{135}{réflexe}
\don{Invoque élément (eau, terre, feu, vent)}{Le joueur invoque une quantité d’un mètre cube d’eau, de terre, de feu ou d’air (un seul à la fois). Cela ne peut pas servir d’arme.}{Vigeur + Survie + Pureté}{aucun}{112}{instantané}

\clearpage
\section{Cahalithe}
\begin{description}
\item[Nom:]{Mathew LittleHawk}
\item[Origine:]{USA}
\item[Auspice:]{Cahalithe}
\item[Tribu:]{Os de l'Ombre}
%\item[Os:]{Perpetuel Gamin}
%\item[Sang:]{La clé de voute}
%\item[Attache Spirituelle:]{Un dream-catch de ta tante}
%\item[Attache Matérielle:]{Ton crayon}
\item[Profession:]{Dessinateur/Tatoueur}
\item[Nom de guerre:]{Sad Song}
\item[Age:]{22 ans}
\item[Histoire:]{
Tu es un jeune garçon de la tribu amérindienne des Pikunis, les blackfeets. Tu as toujours été passionné par les arts graphiques: fan de dessin animés, comics, BD et manga. Tu dévores tout ce que tu peux trouver.
Ton enfance est marquée par la précarité de la vie en réserve. Tu as appris à dessiner tout seul, et tu tiens un blog de comics (et un compte sur deviant-art). Tu as fait quelques illustrations de livre, site web en
freelance. Ton activité ne plaît pas à tes parents qui voit ça comme une marque trop profonde d'intégration dans la vie des hommes blancs. Tu partages ce protectionnisme envers ta culture et la méfiance des blancs mais tu sais bien faire la séparation.
Pendant ton enfance, tes parents, ton frère et tes deux sœurs n'ont jamais été de véritables soutiens. Ton frère a voulu prendre les armes pour la cause et est devenu un hors-la-loi/trafiquant de drogue/d’armes… Il purge une peine de prison. \\
Tes soeurs s'en sortent un peu mieux. Elles travaillent à la ville d’Helena (capitale de l’Etat). \textbf{Lucy} est serveuse et danseuse dans une boite de strip-tease. La dernière \textbf{Jessy} est télé-démarcheuse pour diverses sociétés.
Ta tante (Lyana BlackFeather) était le seul soutien dans ta famille, toujours douce et attentive. Elle te racontait souvent des histoires de fou: avec des monstres et des groupes de héros. A l’adolescence, tu as commencé à expérimenter des rêves étranges. 
Tu voyais les histoires de ta grand-mères prendre une couleur et une saveur tout à fait réaliste. Un jour, tu as senti un appel de ta tante. Elle était en danger. 
Tu ne pouvais te l'expliquer mais tu en étais certain. 
Tu as couru le plus vide possible vers la source des appels. Elle et ses amis étaient dans le monde de tes rêves (ou plutôt cauchemar). 
Ils affrontaient un truc monstrueux, un violent esprit. \\
Ta tante était gravement blessée. Ton instinct t'a poussé à l'éloigner de là. Puis, c'est le noir. Elle et sa meute se battait contre un féroce esprit de maladie. Elle a été infectée et blessée.
A cause de ton intervention, elle n'est pas morte avec sa meute. Elle a gagné quelques heures. Elle a en profité pour t'enseigner le secret de cet esprit. Elle t'a dit d'aller voir un loup-garou ermite. D'après elle, c'était le seul dans la région à pouvoir battre cet esprit. Avec beaucoup de difficulté, tu l'as persuadé de t'aider et à vous deux vous avez détruit ce monstre. \\
Puis, tu as réussi à le convaincre de t'aider. Devant sa bravoure et parce que tu connaissais plus aucun autre loups garou, tu lui as demandé de devenir ton alpha. Vous avez crée la meute et récupérer les autres au fur et à mesure.
Tu t'inspires beaucoup des histoires qu'elle te contait pour faire tes dessins. Sous un pseudonyme, tu écris un comics qui marche pas mal, assez pour te mettre à l’abri financièrement.

Avis:\\
\underline{\Glen} : Un vétéran, il envoie du pâté mais il ne semble pas apprécier son poste d'alpha. \\
\underline{\Laura} : Elle conseille beaucoup l'alpha, et semble comprendre les problèmes de la meute. Tu aimerais avoir autant de poids qu'elle dans la meute.\\
\underline{\Andy}: un type assez secret, il ne souhaite pas s'attacher à vous. Il vous cache un secret. Rien de terrible mais il a visiblement peur de vous le dire.\\
\underline{\Kelly}: Elle est traumatisée, et évite le combat le plus possible. Mais quand il faut, bon dieu, elle est là.\\
}
\end{description}
\clearpage
\textbf{\large Dons}
\vspace{0.5cm}

\don{Conscience de meute}{Obtenir des informations la position autres membres de la meute: loin vers le nord, très proche à sur la gauche, ainsi que la forme utilisé par ton frère, activité générale, gravement blessé ou inconscient}{Astuce + Empathie + Sagesse}{aucun}{118}{Réflexe}
\don{Camaraderie}{Calme les troupes, la meute gagne +1 au jet de résistance à la rage mortelle pour la scène, de plus, si une action en coopération est faite les jets de soutien gagne +1}{Manipulation + Persuasion + Sagesse}{Aucun}{117}{Réflexe}
\don{Vision de mort}{voir les troubles laissés par la mort: mort violente et les fantômes}{}{}{127}{}
\don{Témoignage de cadavre}{Le cadavre décrit ce qu'il a vu depuis sa mort et les infos ne peuvent pas être plus ancienne que 24h}{Manipulation + Occulte + Pureté}{1 point d'essence}{129}{Instantanée}


\clearpage

%\item[Os:]{Perpetuel Gamin}
%\item[Sang:]{La clé de voute}
%\item[Attache Spirituelle:]{Un dream-catch de ta tante}
%\item[Attache Matérielle:]{Ton crayon}

\section{Irraka}
\begin{description}
\item[Nom:]{Andy Zeiner}
\item[Origine:]{USA}
\item[Auspice:]{Irraka}
\item[Tribu:]{Maitre du fer}
\item[Profession:]{Comptable}
\item[Nom de guerre:]{Faceless}
\item[Age:]{28 ans}
\item[Histoire:]{
{\footnotesize Ton véritable nom est Christopher Malone. Tu es né dans le Bronx (NYC), dans une famille assez pauvre. Tu as bossé comme un taré pour te sortir de là. Finalement, tu as obtenu un diplôme universitaire en
Comptabilité / Gestion / Droit des affaires. Le job parfait, tu as mené la belle vie une paire d'années, puis cela s'est compliqué. Dans tes clients, il y avait beaucoup de trafiquant en tout genre qui
cherchaient à laver leur argent sale.
Quand tu t'en es rendu compte, tu as vu l'opportunité de faire payer à ses types ce qu'ils ont fait à ta communauté pendant des années. Tu leur as volé de l'argent, en grosse quantité.
Quand cela a commencé à se voir, tu as reçu de grosse menace. Par chance, Riley Sulliven une agent du FBI t'a arrêté pour fraude fiscale. Tu as pu négocier avec elle pour donner tes anciens clients sur
leur trafic d'armes/drogues/filles et bien d'autres en échange de ton amnistie.\\
Tu es donc rentré dans le programme de protection des témoins du FBI. Dans un motel sous bonne garde, tu as été attaqué. Un des agents du FBI avait trahi ta position pour un petit paquet de fric.\\
Les deux autres agents avait été tué et le traître s'était auto-blessé pour donner le change. Quand les hommes de mains ont voulu jouer avec toi en te jetant dans un bac d'acide. La douleur a réveille cet instinct de prédateur en toi. Le lendemain, tu étais nu au beau milieu de quatre cadavres.\\
 Une meute t'a trouvé et t'a expliqué le monde qui s'offrait à toi.
Tu as vite compris que les grandes villes, c'était fini pour toi. Tu te ferais repérer trop rapidement. Tu as accompli ta vengeance en tuant l'agent du FBI avec l'aide de Riley Sulliven.\\
Elle reste la seule personne qui connaît ta couverture. Vous avez peur qu'il y ait d'autres taupes au FBI. Elle se demande encore comment tu as réussi à survivre à l'enlèvement. \\
Après avoir accompli ta vengeance, tu as du faire profil bas, car même en prison, les types que tu y as envoyé reste puissant.\\ Tu es donc parti loin de New York, dans l`Amérique profonde, tu as trouvé une meute.\\
Cela fait maintenant 2 ans que tu traînes avec eux. Tu ne leur as jamais dit ton passé. Pour eux, tu es Andy Zeiner et cela commence à te plaire. De l'eau à couler sous les ponts, tu peux te montrer moins paranoïaque. Les mecs qui t'en voulaient sont en prison et tu es un loup-garou maintenant et ta meute sera là.\\
Tu vis une relation assez compliquée mais fusionnelle avec Riley Sulliven. }


Avis:\\
\underline{\Glen} :  C'est ton alpha, il est cool. mais il ne semble pas aimé avoir des responsabilités. Est-ce normal pour une meute d'avoir un alpha qui ne veut pas l'être ?\\
\underline{\Mathew} : Un jeune idéaliste, il est toujours motivé pour tout. C'est fatiguant.\\
\underline{\Laura} : elle est loin d'être conne mais la poisse l'aime beaucoup. Presque autant que toi. \\
\underline{\Kelly}: Tu ne comprends pas, elle pourrait faire une super alpha, et rouler sur tous vos ennemis, mais elle évite le combat le plus possible.\\
}
\end{description}
\clearpage
\textbf{\large Dons}
\vspace{0.5cm}

\don{Nuit Noire}{coupe les lumières électriques dans une zone (\begin{math}200m^2\end{math} par succès)}{Astuce + Larcin + Ruse}{1 volonté}{146}{Instantanée}
\don{S'esquiver}{Se libérer de liens (menottes, camisole...)}{Aucun}{1 point de volonté}{132}{réflexe}
\don{Poudre aux yeux}{Permet de faire oublié sa présence: «ce ne sont pas ces droides la que vous recherchez.»}{Manipulation + Subterfuge + Honneur (vs Calme + Instinct Primal)}{}{111}{}
\don{Faire le mort}{Passe pour mort, un jet en Intelligence + médecine doit être fait et dépasser le score de l'activation du pouvoir. Le corps reste mort pendant 24h ou jusqu'à que le LG décide de se réveiller. Il reste conscient de ce qu'il se passe autour de lui.}{Astuce + Subterfuge + Ruse}{1 volonté}{111}{Réflexe}

\clearpage
\section{Rahu}
\begin{description}
\item[Nom:]{Laura Bennett}
\item[Origine:]{USA}
\item[Auspice:]{Rahu}
\item[Tribu:]{Seigneur des tempêtes}
\item[Profession:]{}
\item[Nom de guerre:]{blood queen}
\item[Liste de dons]:
\item[Age:]{22 ans}
\item[Histoire:]{
{\footnotesize Tu es une jeune fille issue d'une famille tout à fait classique, tu as une sœur jumelle (Elizabeth). Elle a toujours été la préférée de tes parents. Elle était douée à la école; pas toi. Toi, tu te bagarrais tout le temps, souvent
pour l'aider mais c'est toi qu'on punissait.
Vous étiez très proche même si les adultes t'éloigner d'elle car ils pensaient que tu aurais une mauvaise influence sur elle. Le 22 avril 2000, la police a sonné chez vous, pour vous annoncer la mort de vos parents.
Vous n'avez que 10 ans et vous êtes rentré dans le système d'aides sociales. Ta sœur s'est enfermée dans les études et a trouvé une famille d'accueil assez facilement. Pour toi, ce fut les centres d’accueil puis de redressement.\\
Plusieurs années après elle a réussi à te retrouver. Elle a essayé de t’aider mais la drogue a été la plus forte.  Tu as fait des trucs vraiment sale pour tes dose: vols, agressions et meurtres.\\
Un soir, tu négociais auprès de ton dealer ta dose. Il refusait de te lâcher la dose pour le peu d'argent que tu avais. Les flics vous sont tombés dessus. Tu as profité de la confusion pour voler le stock du dealer.\\
Tu t'es payée un gros trip, tu en avais jamais eu autant. Tu savais que de toute façon, tu ne survivrais pas à cette "fête" mais t'y as survécu et les mecs ont commencé à chercher leur produit volé. Sans le stock, la police
n'a pas pu les mettre en prison. Ils ont donc été libéré rapidement. Quand ils t'ont retrouvé, ils ont jouer avec toi et des barres de fer. Cela casse vite des os humains.\\
Le lendemain, tu t’es réveillé dans une chambre de la maison de ta sœur. Son mari était en voyage d'affaire. Elle t'a expliqué que ta rage s'était exprimée sur ton dealer et son pote. Tu es donc une loup-garou comme elle. Le bon coté, c'est que devenir loup garou t'a rendu complètement insensible aux sirènes de l'héroïne. \\
Elizabeth t'a fait rentrer dans sa meute. Tu as été accepté non sans difficulté mais tu as vite remarqué le manque d'imagination et la peur d'agir de ses membres. Tu t'es imposée en quelques mois comme l'alpha de la meute.
Un jour un plan à merder quelques parts, et ta sœur s'est retrouvé seule face à des flics ripoux qui foutaient la merde sur votre territoire. Elle les a tué. Le problème, c'est qu'elle a montré sa vrai forme devant des humains. La \textbf{lubie} l'a protégé mais elle a été arrêtée par des flics survivants. Des caméras de surveillances la montre rentrer sur les lieux du crimes et en sortir quelques secondes après couverte de sang.
Ta meute a voulu l'exclure. Tu savais qu'elle ne supporterait pas la prison et encore moins d'être renier par ses frères de meute. Tu as porté le chapeau devant ta meute et devant les autorités.
Ta sœur fut libérée et tu as fini en prison. Avec ton casier judiciaire et le meurtre de policier le verdict tomba vite: peine de mort par injection. \\
Tu aurais accepter ce sort sans angoisse. Ce n'est pas comme si tu étais attachée à la vie. Le problème, une injection létale aurait eut autant d'effet sur toi qu'un verre d'eau. Du coup, pour éviter de violer l'harmonie, tu as décidé de t'évader. Ta sœur t’a aidé, ce fut assez facile. L'avantage d'être un loup-garou. Tu as traîné un peu avant de trouver un coin tranquille pour vivre. Une meute t'a accepté. Ta sœur et sa meute ont déménagé de Chicago vers Helena pour être proche de toi. Sa meute a pris le nom de Whispers \& Ashes une fois arrivée. Vous deux êtes les seules personnes à connaître la vérité. Vous vous voyez de temps en temps. Officiellement la police te cherche à l'étranger. La situation devient peu à peu gérable. Ta vrai fausse identité t'aide pas mal. Tu travailles dans le parc animalier privé (celui de ton alpha).\\
Ton alpha veut aller en ville, tu essaieras de rencontrer ta sœur.}

Avis:\\
\underline{\Glen} :  Un bon alpha, tu l'aides un peu et le conseille. Il est mort intérieurement, son poste d'alpha lui redonne un peu la vie.\\
\underline{\Mathew} : Un jeune, il est un peu trop enthousiaste, mais c'est un bon type.\\
\underline{\Andy} : Quelqu'un de très secret, il ne vous fait pas confiance visiblement. Il est volontaire, mais dès qu'il est concerné, il devient transparent.\\
\underline{\Kelly} : Un bon élément, elle fera une très bonne alpha, elle a encore à apprendre et elle doit évacuer sa peine.\\
}
\end{description}
\clearpage
\textbf{\large Dons}
\vspace{0.5cm}

\don{Clarté}{+ 5 en init}{Aucun}{1 Point d'Essence}{136-137}{Réflexe}
\don{Brouillard silencieux}{Créer un brouillard dense, 20 mètre carrée par point d'instinct primal, règle à couvert pour toutes actions de combat, la meute du LG voit à travers}{Manipulation + Survie + Ruse}{1 essence}{123}{Instantanée}
\don{Hobbling Gaze}{En insinuant la peur dans l'esprit d'une cible. Elle ne peut pas fuir à toute vitesse. Réduit la vitesse de l'adversaire de 1 par succès.La réduction de vitesse fonctionne 1 minutes par succès.}{Dextérité + Sport + Honneur- Résolution de la cible}{1 essence}{123}{Instantanée}
\don{saut puissant}{+6 dés pour faire un saut}{}{}{116}{}

\clearpage
\section{Ithaeur}
\begin{description}
\item[Nom:]{Kelly Clark}
\item[Origine:]{Sioux}
\item[Auspice:]{Ithaeur/Shaman}
\item[Tribu:]{Griffe de Sang}
\item[Profession:]{Médecin Militaire}
\item[Nom de guerre:]{911}
\item[Age:]{34 ans}
\item[Histoire:]{
Tu es une jeune femme de 34 ans issue d'une famille amérindienne. Tu es née dans une réserve indienne. Tu as toujours été une élève studieuse. Pour réaliser ton rêve de devenir médecin, tu as accepté un
programme de bourse pour aider les gens des réserves indiennes. L'armée américaine payait tes études et tu devais juste faire quelques années dans l'armée. Cela semblait un bon plan sur le papier. Tu as
signé ton contrat, le lundi 10 septembre 2001.
Tu as fait plusieurs tours en Irak et en Afghanistan. De cette expérience, tu restes assez traumatisée. La traumatologie de guerre, c'est vraiment un truc sale. Tu étais le médecin d'un régiment des
marines.
Tu as gagné le titre de "Luckiest Kelly" dans ton régiment, tu as souvent pressenti les problèmes. Tu étais la mascotte, l’élément non-combattant qui fallait protéger mais aussi l'ange gardien des soldats.
Ce sens du danger était principalement due à ton sang de loup, tu en as conscience maintenant. C'est d'ailleurs lors d'une mission en Afghanistan que tu as expérimenté ton premier changement.
Votre section avait pris position en haut d'un plateau, l'autre section de votre régiment avez pris position sur un plateau voisin légèrement plus bas dans la vallée.
Durant la deuxième nuit de surveillance, vous avez compris que l'autre section était sous un feu nourri. Après la première vague d'attaque, les communications radios annonçaient que Peter Allister avait été touché.
C'était le médecin de l'autre section et ton petit ami (pas tout à fait officiel de l'époque). Tu étais si inutile alors que des gens avez besoin de toi. Cela t'a mis dans une rage folle. Tu as décide de rejoindre l'autre plateau à 10km de là, à pieds. C'est complètement idiot mais une chose en toi disait que tu pouvais le faire. Tu ne te rappelles pas de la suite.
Le lendemain tu t'es réveillée dans une maison d'un village de la vallée que vous deviez surveillez. Un loup garou afghan t'a expliqué les règles qui venait avec ta nouvelle condition.
Lui et sa meute ont attaqué la section Charlie avec des talibans sous leur ordre. Ils étaient clairement hostiles aux américains mais le fait d'être loup-garou t'a sauvé la vie. Tu ne veux pas savoir ce qu'ils aurait fait à une femme ennemi.
Ils t'ont renvoyé à ta base. Ils ont été clair: soit tu rentres au pays, soit la prochaine fois qu'il te croise, tu seras un loup-garou essayant de leur voler leur territoire. Bref, après ce mois passé chez l'ennemi, la hiérarchie et ton régiment n'avaient plus confiance en toi.Tes études étaient finies et tu avais presque fait ton temps dans l'armée.
Tu es restée deux mois à travailler à l'hôpital de la base internationale.
Tu es rentrée et tu as essayé de trouver un boulot et une meute. Le boulot, tu en as chié. Ta réputation au sein du milieu médical était assez détruite. Tu aides dans les réserves mais cela te fait clairement chier. Tu revais de grandes choses, et te voilà à soigner des rhumes et les blessures d’après bagarre. Tu as le sentiment d’être utile mais ton talent est gaspillé. Parmi les habitant de la réserve, tu as trouvé une meute.
Ce week-end à Helena te donnera peut-être des opportunités.

Avis:\\
\underline{\Glen} : Un bon alpha, il t'occupe car il sait que tu en as besoin. \\
\underline{\Mathew} : un jeune idéaliste malgré les morts et la violence des Urathas, il est plein d'energie. C'est fatiguant. \\
\underline{\Laura} : Elle semble s'être porté volontaire pour avoir une vie de merde. \\
\underline{\Andy}: Un type assez secret, il ne vous fait pas confiance mais il est là quand il faut agir. \\
}
\end{description}
\clearpage
\subsection*{Dons}
{\footnotesize
\don{Œil sur les deux mondes}{Permet de voir dans les deux mondes à la fois}{Astuce + occulte + Sagesse}{aucun}{104}{Instantanée}
\don{Vitesse de Père-loup}{Double la vitesse, la vitesse est doublée tant que l'utilisateur court. S'il s'arrête, il doit dépenser à nouveau le coût du don.}{aucun}{1 essence}{}{Réflexe}
\don{Déchirer le métal}{Il est plus facile de couper le métal. Ignore autant de point de rigidité que de succès obtenu pour l'activation du don.}{Force + Artisanat + Gloire}{1 essence}{116-117}{Réflexe}
\don{Coup Fracassant}{Les dégâts contondants avec tes poings deviennent des blessures létales (en forme humaine)}{}{1 point de volonté}{116}{}

\subsection*{Rituel} 
\rituel{Rite Funéraire}{Rend hommage à un LG mort. Sorte de cérémonie d'enterrement pour la société loup-garou.}{Etendu, 10 succes, chaque test représente 15 minutes.}{Le rite prend 1h environ. Chaque participant verse un peu de leur sang sur le défunt et sur le lieu de repos éternel: tombe, bucher, océan. Chaque participant hurle son deuil}{151}{1}

\rituel{Invoqué un Gaffling (esprit aussi puissant qu'un loup-garou).}{Fait un cercle au sol et s'assoie au milieu, une offrande est faite pour appater l'esprits. La nature de l'offrande depend du type d'esprit désirer. Le ritualiste peut invoque un esprit particulier s'il connait son nom ou un type d'esprit. Les esprits n'aiment pas être invoqué.}{Étendu, 40 succes, chaque jet correspond à 1 minutes.}{Un esprit du type demandé vient et le rituliste doit conclure un accord avec l'esprit pour obtenir des infos ou services.}{152}{2}

\rituel{Rite de Guérisson}{Guérit les dégats aggravées(pour un maximum de 5). Il faut dépenser 2 points d'esssence pour soigner 1 point. Soit le ritualiste depense les 2 soit, moitié-moitié entre le ritualiste et la victime. }{Etendu, de 5 à 25 succes, chaque test represente 15 minutes.}{Les blessés sont regroupés en cercle autour du ritualiste. Il invoque les esprits des ancêtres et ceux de la force et de la compassion. Nettoyage des blessures: verser de l'eau ou lècher les blessures.}{161}{3}
}
\chapter{L'histoire commence}
\section{Chronologie des événements}
\begin{itemize}
\item j-20, un petit tremblement de terre, la maison de Casey Stengel s’effondre. Les secours arrive et le sort des décombres après plusieurs heures. Sa femme est retrouvée morte. Son fils jouait devant la maison, quand son père Casey retrouva les esprits, il fallut se rendre à l’évidence son fils avait disparu. La police d’Helena a lancé les recherche, en pensant que l’enfant avait juste fuit. 
\item j-19, le père utile ses pouvoirs pour revenir dans le temps afin de sauver sa femme et son fils. Cela a échoué et il a rencontré une créature des abysses. 
\item j-18, la police abat un homme après une course poursuite en voiture. L’homme a forcé un barrage et tué deux policiers. La police trouvera des affaires personnelles du petit Stengel dans la voiture. 
\item j-17, la police trouve le corps de l’enfant, dans la forêt proche du barrage routier.
\item j-15, la créature des abysses et les récents événements rendent complètement fou Casey Stengel. Une part de lui même souhaite prévenir des événements similaires d’arrivée à quelqu’un d’autre. Il se sent coupable de la mort des deux policiers et reconnaissant de leurs actions.
\item j-14, il joue au loto et gagne une énorme somme d’argent, en utilisant sa magie.
\item j-13, \textbf{Casey Stengel} commence à enlèver ses cobayes: des vampires et des loups-garous. 
\item j-12, \textbf{Casey Stengel} continue d’enlèver ses cobayes: des vampires et des loups-garous.
\item j-11, \textbf{Casey Stengel} continue d’enlèver ses cobayes: des vampires et des loups-garous.
\item j-10, \textbf{Casey Stengel} sauve \textbf{Jessy} (la sœur de \Mathew) en kidnappant une vampire qui voulait goûter du sang de \textbf{Sang-de-loup}.
\item j-9, \textbf{Casey Stengel} enlève des mages, des amis à lui qu’il a invité à Helena en reprenant contact avec eux.

\end{itemize}
\section{Information de départ}
Après avoir décris l’univers de Loup-Garou, il est important de laisser l’alpha expliquer pourquoi, ils vont aujourd’hui à Helena (capitale de l’État du Montana). Les joueurs ont le planning public du gouverneur. En fin de matinée, il fera un discours au musée tenue par la NRA. 

\section{Avant la ville}
Les personnages sont dans une vieille décapotable américaine, Les \textit{Creedance clearwater Revival} crache leur mythique \textbf{Fortunate Son} dans l’autoradio.\\
C’est le moment de faire la traditionnelle tournée de la description des personnage. \\
Vous pouvez profiter de ce moment pour lancer un flash radio qui donnera des indications sur le tremblement de terre et les quelques victimes mais également sur les événements dans la ville de la journée. Il faut noyer le tout dans d’autres informations: sortie cinéma, concerts, sport.\\
Le but des personnages est de se rapprocher du gouverneur. Ils sont au courant de son emploi du temps du week-end, le but est d’essayer de parlementer avec lui et  voir s’il est louche. S’il l’est alors il faudra prendre des mesures, sinon il faudra le convaincre d’abroger sa loi sur la chasse.


\section{Arrivée en ville - Musée de l’armement}
Quand les joueurs arrivent en ville, le Gouverneur s’apprête a donné un discours pour inaugurer l’arrivée d’une relique du passé de l’État dans le musée des armes à feu tenu par la NRA (National Rifle Association).
Il faut imaginer une scène avec des invités, des journalistes, un public qui sera tenu à l’écart. Les joueurs arriveront bien une heure avant le début de la «fête». Bien sûr, il y aura son lot de manifestants: pro-gouverneur, pro-life, pro-républicain et des opposants: pro-LGBT, Black lives matter, pro-IVG, pro-écolo.\\
La conférence de presse a lieu dans le jardin du musée ou un char Sherman de la seconde guerre mondiale est entreposé.\\
Les joueurs peuvent essayer de se rapprocher le plus prêt possible de l’estrade, en utilisant leur relationnel ou leur talent pour arriver dans le carré des invités: \roll{Séduction: Présence + Persuasion}, \roll{Corrompre: Manipulation + Persuasion} ou \roll{gros mensonge: Manipulation + Expression} \\ 
Bien sur, la sécurité autour du musée est assurée par la Helena Police Department. \\
La voiture du gouverneur arrive proche de l’estrade des intervenants. Le directeur du musée annonce au micro son arrivée, mais il tarde à sortir. Le service de sécurité du gouverneur (des fédéraux) se déploie entre la voiture et l’estrade pour protéger le parcours.\\
C’est à ce moment qu’un coup de feu est entendu et Bradley J. Kruger (Western Regional Director de la NRA et directeur d’une associtation locale de chasse) s’effondre.\\ La voiture du gouverneur s’en va très rapidement et un mouvement de panique se déclenche dans la foule.\\ 

\section{Musée de l’armement - enquête}
Si \Kelly a activé son don l’œil sur les deux mondes, (il n’est pas recommandé de le faire dans un contexte social), Elle peut voir toute la scène dans l’hisile. 
\subsection{Le meurtre}
\label{meurtre}
Elle verra donc un loup-humanoïde entouré de fumée noire qui a tiré un coup de feu dans le dos de l’invité: Bradley J. Kruger puis le monstre s’écarte et disparaît au niveau de la voiture du gouverneur.\\
Si elle le fait trop tard, elle verra un tas de fumée noire disparaître au niveau de la voiture du gouverneur.\\
\subsection{L’après meurtre}
La première chose à faire est de gérer le mouvement de foule: \roll{Avancer contre la foule: vigueur + athlétisme} ou \roll{Se faufiler dans la foule: Dextérité + furtivité}.\\
En fonction de leur placement initial et leur succès, ils peuvent atteindre le corps de la victime ou d’autres objectifs. \\
Une fois le calme revenu, les personnages ayant des compétences en arme à feu peuvent se rendre compte que le tir était à bout touchant (vu le bruit) \roll{Identification du tir: Intelligence + Arme à feu}.\\
Les personnages ont tous ressenti un grand sentiment de colère, vengeance et haine à l’instant précis du coup de feu.\\

La zone va être évacuée rapidement par les forces de l’ordre.\\ Il n’est pas possible de passer dans l’hisile proche du musée car il n’y a pas de locus. Cependant, il en existe un pas trop loin. \roll{Trouver un locus: astuce + survie} \Andy bénéficie d’un bonus de +2 dés car il est Irraka. \\

\section{L’attente bac à sable}
Les personnages ont quelques heures à tuer avant la prochaine opportunité de rencontrer à nouveau le gouverneur. Il est possible que les joueurs aient envie d’appeler les contacts dans la ville.
Essayer tant que possible de limiter le nombre d’événements. Deux me semblent pas mal. Si les joueurs font plus ils arriveront en retard pour l’inauguration de la clinique et la gala qui la suit.

\subsection{Rencontre avec les sœurs de \Mathew}
C’est samedi en milieu de journée, elles ne travaillent pas. Les joueurs les trouves chez elles.  
Si les joueurs demandent pourquoi elles n’ont pas répondu aux appels de \Mathew . Elles répondent que le tremblement de terre a pas mal endommagé le réseau de la ville mais 
\textbf{Jessy} dit qu’il n’y a pas que cela. \textbf{Lucy} lui dira de se taire. Les joueurs comprendrons bien qu’il y a un malaise. En insistant un peu \textbf{Lucy} crache le morceau.\\
Elle a été victime d’une tentative d’agression. \\
Elle raconte les faits suivants:\\

\textit{\\
Je rentrais chez moi après le boulot. J’ai senti une présence derrière moi. J’ai accéléré le pas, mais la personne aussi. J’ai couru, et une voie féminine et grave derrière moi dit en riant: «Tu peux courir ma jolie, mais je vais te rattraper», et la personne s’est mise à courir également. Je l’ai entendu me rattraper très rapidement. Quand je l’ai cru sur le point de m’attraper, je me suis retournée en vidant mon stray au poivre. J’ai vu mon agresseur figé sur place comme pris dans la glace. Mon agresseur était bien une femme, dans un horrible cosplay de vampire. Les dents étaient bien faites. Un homme m’a dit que je pouvais partir tranquillement qu’il allait s’occuper de ça. Je suis partie très vite, tu t’en doutes.\\}
\bigbreak

Elle expliquera ensuite qu’elle n’arrive pas à savoir si ce qu’elle vu était vrai ou juste le fruit des valeurs qu’elle respire sur son lieu de travail.
Elle n’a pas vu le type qui l’a sauvé. \\

L’agresseur était bien une vampire. Les joueurs peuvent se poser la question de la présence vampirique à Héléna \roll{Activités vampiriques: Intelligence + occulte }.\\
Effectivement, il y a des vampires à Helena. Une entente cordiale existe entre eux et les loups-garous.\\
Elle a été définie il y a bien longtemps et les loups-garous punissent volontiers les vampires qui ne la respecte pas. Les Whispers \& Ashes ont la tâche de faire respecter cet accord ancestral (principalement parce que les vampires sont sur leur territoire).\\ 

\subsection{Riley Sullivan et le FBI}
\Andy peut essayer de contacter Riley sa copine pour avoir des infos. Elle est débordée par tout le bordel avec «l’attaque sur le gouverneur au musée» et la mort de Bradley J. Kruger (Western Regional Director de la NRA et directeur d’une associtation locale de chasse). 
Riley est une bonne flic, elle posera des questions et rencontrera \Andy et ses copains dans un endroit tranquille. Elle ne souhaite pas que ses collègues apprennent la situation. Cela pourrait être dangereux pour \Andy et pour elle. \\
Lacher des infos, il y a pas de problème. Elle donnera le nom de la victime, que le coup était bien «bout touchant» (les gaz de la combustion de la poudre ont fait des dégât sur le corps de la victime).\\
Les associations pro-NRA et chasse accuse les écolo qui défende eux-même une théorie du complot.\\ 
Si les joueurs souhaitent rencontrer le Gouverneur grâce à elle. Il faudra le justifier. Les personnages et surtout \Andy doivent lancer quelques arguments et infos pour qu’elle accepte de les faire rentrer à la soirée d’inauguration de la clinique Saint-Ann. \\
Elle sera curieuse de savoir pourquoi ils veulent des infos sur le meurtre ou pourquoi ils veulent rencontrer le gouverneur. \\
Elle n’est pas idiote, elle sait que \Andy lui cache des choses depuis son enlèvement. Elle essaiera de lui faire cracher le morceau (sans se douter qu’il est un loup-garou).\\

S’ils arrivent à la convaincre, elle leur donnera rendez-vous dans une rue derrière la clinique vers 20h30 (une demi-heure après le début). Elle les fera rentrer par la porte de service à condition qu’«ils ne foutent pas le bordel».\\


\subsection{La maison d’Elizabeth Bennett (sœur de \Laura)}

La maison est vide. Il n’y a personne. Les affaires des enfants ne sont pas là, ni celle du mari. La voiture n’est pas dans le garage. Il n’y a aucune trace de violence. \\ Ils est possible de retrouver dans l’ordinateur \roll{ Astuce + Informatique} qu’Elizabeth a lu des article sur des disparitions: Comment mener une enquête, comment la police le fait…\\
Il y a un message sur le répondeur de la maison. «Allo! chérie, on est bien arrivée chez ma mère. J’espère que cela va pour toi. Appelle-moi quand tu auras réglé le problème avec tes «collègues».\\
Il fait très chaud dans la maison. Tellement chaud, les personnages transpirent vite et ont du mal à respirer.\\
S’il regarde dans l’hisile, ils verront la maison complètement recouverte de flammes. 
Le totem de la meute des Whispers \& Ashes est un phénix (esprit de feu). Il est complètement fou car sa meute n’existe plus. Il ne la ressent plus. Son lien est brisé. \\
La maison est un locus à cause de l’esprit qui consume la frontière entre les deux mondes.\\
Il est compliquée de le calmer et de lui parler \roll{le calmer : Présence + intimidation} ou \roll{lui parler : Présence + empathie + instinct primal}. \\
Les joueurs peuvent se battre contre lui pour le calmer ou \Glen peut utiliser son pouvoir d’élément pour attirer l’attention de l’esprit.\\
Il est possible de comprendre son état \roll{ Astuce + occultisme}.\\
L’esprit expliquera que sa meute est partie. Il ne la sent plus.\\ Sa meute enquêté sur la disparition de «vampires».\\
S’ils posent la question, les vampires ont leur elysium en ville dans un temple (maçonnique). L’esprit leur donne des indications pour le trouver en ville (à proximité de l’école Sainte-Héléne). C’est facile à trouver en ville.\\
Il demandera à la meute de trouver la sienne. Il ne pourra pas les aider: trop affaiblit et trop instable.\\

\subsection{Visite de l’hisile}

Le monde des esprits est un vaste bordel. Les esprits sont en roue libre. Cela témoigne d’une absence des loups-garous depuis un certains nombres de jours.\\
Les joueurs peuvent négocier avec un esprit (il y a de tout: chien, lampe, émotions: inquiétude, sécurité, joie …). Cela dépend ou ils outrepassent.\\
Les esprits vont se moquer d’eux: Vous allez bientôt disparaître comme tous les autres, Les lupus sont une espèce en voie d’instinction, par ici. Vous voulez nous calmer vous êtes que 5. Etc.
Ils comprennent quand même que cela fait 12 jours environ que les meutes locales n’ont pas été vu dans l’hisile.\\
S’ils sont proches du musée ou s’ils pensent à y aller dans l’hisile. Ils peuvent parler avec l’esprit du sherman. Il demandera soit 5 points d’essence, soit que les joueurs détruise un mur à côté de lui qui lui gâche la vue (Il faut jouer le sherman comme un vieux de la vieille qui connaît la guerre quoi). L’esprit décrira la scène du meurtre (voir \ref{meurtre}). Il ira de son commentaire: «Ce n’est pas une façon de faire la guerre. Une histoire personnelle moi je vous dis.».\\


\subsection{Pas d’idée ?}
Si vos joueurs n’ont pas d’idées, tacher de faire attention qu’ils entendent le nom de la victime et son rôle (Bradley J. Kruger - Western Regional Director de la NRA et directeur d’une associtation locale de chasse).\\
Puis, passer à la suite rapidement.


\section{Inauguration de la clinique}

\section{Le gala}


\section{Rencontre avec les créatures}

Les créatures sont des mélanges entre humain, vampire et loup garou. 

\section{Le conflit moral}

Le mechant leur propose un marché. 
Il garde les loups garous en otage en échange il fournit des soins à l'ensemble de la ville.


\end{flushleft}
\end{document}

