% !TEX encoding = UTF-8 Unicode

%!TEX TS-program = xelatex
%!TEX encoding = UTF-8 Unicode

\documentclass[oneside,12pt]{book}
\usepackage[left=2cm,top=1cm,right=3cm,nofoot]{geometry}                % See geometry.pdf to learn the layout options. There are lots.
\geometry{a4paper}                   % ... or a4paper or a5paper or ... 
\usepackage{tabularx}

\usepackage{fontspec,xltxtra,xunicode}
\defaultfontfeatures{Mapping=tex-text}
\usepackage[french]{babel}
\usepackage{listings}
\usepackage{graphicx}

\title{L'héritier}
%\subtitle{Scénario Imperium}
\author{Renaud "ObiWan" Guezennec}
\date{}

%\let\origdescription\description
%\renewenvironment{description}{
%  \setlength{\leftmargini}{0em}
%  \origdescription
%  \setlength{\itemindent}{1em}
%}


\begin{document}

\maketitle \clearpage
\tableofcontents \clearpage

\begin{flushleft}
    \chapter{Introduction}
        \section{Introduction à L'univers de Dune}
         \section{Glossaire de l'univers et du Scénario}
\begin{itemize}
\item[Les Grandes Maisons]
\item[Les Maisons Mineures]
\item[La Guilde]
\item[Le Bene Gesserit]
\item[Le Landstraad]
\item[Le CHOM]
\end{itemize}

\section{Personnage}
\begin{description}
\item[Alto Richèse]{ Prince héritier}
\item[Margus Richèse]{  Oncle au pouvoir }
\item[Vaness Richèse]{  Mère du prince, amante de l'oncle et Femme de feu  }
\item[Salmus Richèse]{  Duc mort en début d'histoire.}
\item[Phillipus Veyron]{ Intendant de la maison Richèse, Père d'un PJ.}
\item[Vanlis Teozys] { Comte à la tete de la famille Teozys, Amis personnel de Salmus Richèse. Vivant sur la planète Karkall}
\item[Stella Teozys] {Fille du comte Vanlis.}
\end{description}
\chapter{Personnages Joueurs}
\clearpage
\section{L'esclave}
\begin{description}
\item[Nom:]{Aly'o Erigann}
\item[Origine:]{Inconnue}
\item[Profession:]{Esclave au service du prince}
\item[Age:]{19 ans}
\item[Histoire]{ 
Tu ne connais pas ton passé, du moins, tu ne souhaites pas t'en souvenir. 
Ton premier souvenir évoque un grand sentiment de solitude et de peur. Tu te souviens avoir été recueilli à ce moment par un homme lugubre. Quelques mois plutard tu as été vendu à la famille Richèse. Tu es devenue l'esclave personnel du Prince "Alto Richèse". Il te traite assez bien. Tu manges à ta faim. Tu n'es pas battue. Encore mieux, il laisse parfois un peu d'Epice pour toi. Tu es légalement obligée de faire tous ce qu'il te dit. Il t'a promis de te libérer un jour. Tu n'y crois pas trop mais bon
une part de toi souhaite y croire. Ton quotidien est assez épuisant: Tu dois te lever avant lui, l'aider à s'habiller, gouter tous les plats avant lui, le laver et surtout faire tout ce qu'il lui passe par la tête. 
Enfant, il était un peu cruelle avec toi. Maintenant, il est beaucoup plus raisonnable et c'est devenu beaucoup moins difficile. 
Cela fait 2 ans que vous avez quitter la planète de la famille Richèse () pour la planète Karkall. 
Elle est le domaine de la famille Teozys. En effet, le Duc Richèse (père de Alto) l'a envoyé pour être à l'abri des assassins et autres complots. 
En même temps, il y reçoit une éducation très poussée. La famille Teozys souhaite obtenir les faveurs d'Alto quand il sera au pouvoir et probablement obtenir un marriage. 
Tu as bien vu les regards de Stella (Fille du comte Teozys). Le comte s'arrange souvent pour que les deux jeunes gens se rencontre. \\

 Depuis le temps, tu l'aimes tendrement ton prince.  
Il est évident que dans sa position, il ne t'accordera guère plus d'intérêt qu'un animal de compagnie mais tu es bienveuillante à son égard. 
}
\end{description}
\begin{itemize}
\item[Alto Richèse]: personne que tu respectes beaucoup, il a appris à écouter tes conseils. Cela t'amuse comme il formule les choses pour croire que l'idée venait de lui. Il lui arrive de faire des grosses colères jamais tourner vers toi. Il t'a plusieurs fois parler de la mort et du suicide. Cela t'inquiète un peu. Il devient de plus en plus absorber par les problèmes du pouvoir. 
		\item[Edward Linoilis: le maitre d'armes]: Bien trop, inbu de sa personne, il mangerai du cirage pour briller en société. Il ne t'accorde pas la parole. Tu as reçu quelques coup de pied de lui. Avant qu'Alto le rappelle à l'ordre. Ils s'entrainent tous les deux 4 heures tous les matins. Tu assistes souvent à l'entrainement. Alto n'est vraiment pas douée pour la voix des armes. 
		\item[Penwyr Lanton : le prof particulier]: Un gentil bonhomme, un peu rondouillard. Il est clairement un grand intellectuel, et un fin manipulateur. Il ridiculise volontier les arrogants. Il a souvent une petite attention en vers toi: un sourire ou un clin d'oeil. Il a clairement des habitudes qui te fait penser qu'il a connu la pauvreté. D'après le peu que tu sais, Il a participé au rétablissement de l'ordre sur la planète KarKall, et a démontrer ses capacités de stratèges. Il a pu monter en grade. Après cet évènement, il a préféré suivre la voie du savoir. C'est un expert en histoire et en droit. 
		\item[Octave Siav: Professeur de pilotage ]: Un peu trop casse coup à ton gout, il pense souvent à faire plaisir à son élève pilote mais peu à ses passagers. Il n'est pas très bavard sur lui, tu ne sais pratiquement rien de lui.  
		\item[Alécia Veyron : Courtisanne ] C'est la fille de l'intendant de la maison Richèse, Elle vit la bas. Elle est la candidate la plus avancé pour l'union avec Alto sauf si son père arrive à l'unir avec une personne d'une famille plus puissante. Elle est très méchante avec toi. Elle déteste toutes les femmes autour de son prétendant. Tu penses qu'elle devrait calmer un peu. 
		\end{itemize} 
\clearpage


\clearpage
\section{Le maitre d'armes}
\begin{description}
\item[Nom:]{ Edward Linoilis}
\item[Origine:]{KarKall}
\item[Profession:]{Maitre d'armes}
\item[Age:]{34 ans}
\item[Histoire:]{
       Tu es un pur enfin de la haute bourgeoisie Karkallienne. Depuis très jeune, tu as démontré une grande maitrise des armes. Tu as participé gagner de nombreux duel pour la famille Teozys. Maintenant tu vends tes qualités à la famille
Teozus (qui règne sur Karkall). Tu es devenu le maitre d'armes officiels de Alto Richèse. C'est un grand honneur pour toi, tu as assez peur de lui. C'est un gentil garçon mais on t'a toujours dit de te méfier des grandes familles. Ils seraient capable des pires attrocités. Le travail t'apporte beaucoup de chose. La première est bien payé. L'honneur est la seconde, tu fais parti des gens qui vivent au chateau du comte, ce n'est pas rien. Tu as accès à la meilleure nourriture de la planète. Tu peux même te permettre de consommer de l'épice, de temps en temps. Tes journées sont rythmées sur les séances d'entrainement: 4 heures tous les matins. L'après midi, c'est repos. Tu en profites pour aller courrir les jupons, en ville. Ton grand maleur, c'est le niveau de ton élève, il est vraiment nul en combat. Cela ne le motive pas. En 2 ans, d'entrainement, il n'a jamais réussi à passer ton bouclier protecteur.     

}
		\begin{itemize}
		\item[Alto Richèse]: Ton élève, une personnalité bien a lui. Il est assez juste et intelligent dans des décisions. Cependant, il ne démontre aucune motivation pour la voie des armes. 
		\item[Aly'o Erigann: l'esclave d'Alto]: Fille plus intelligente qu'il n'y parait, tu aimais la taquiner ou la remettre à ça place mais Alto te l'a interdit (tu as recu 5 coups de fouet pour cela.). Cependant malgrés sont status social, c'est une très belle femme. Il est dommage qu'elle ne soit pas plus respectable.  
		\item[Penwyr Lanton : le prof particulier] : C'est une personne pour qui tu as du respect. Toute la planète connait ses actes de bravoures et ses coups de génie. Il n'est pas exception en combat (surtout qu'il est assez agé) mais il a gagné des batailles alors que tout le monde voyaient cela perdu d'avance. 
		\item[Octave Siav: Professeur de pilotage ]: Tu ne le connais pas trop, il semble avoir un peu le même profil que toi mais il est très discret, et ne sais pas profiter de la vie. Tu aimes le charier la dessus mais il ne montre pas vraiment d'intérêt à te répondre.
		\end{itemize} 
\clearpage
\end{description}

\clearpage
\section{Prof}
\begin{description}
\item[Nom:]{Penwyr Lanton}
\item[Origine:]{Karkall}
\item[Profession:]{Précepteur/Conseiller/Professeur}
\item[Age:]{53 ans}
\item[Histoire:]{
Jeune tu t'es engagé dans les forces de sécurités de Karkall. Lors d'une révolte ouvrière, ton unité fut envoyée. Après différentes escarmouches, tu t'es vite retrouver à commander une dizaine d'hommes. Tu as pu élaborer une stratégie afin de mater cette révolte et découvrir les véritables protagonistes de cette histoire. Une famille adverse avait lancer cette révolte pour déstabiliser la famille Teozys. Cette grande réussite, te permit d'obtenir certains privilèges de la part du Comte Teozys. Tu choisis alors de suivre les enseignements sur la politique,le droit et l'histoire. Tu as réussi à changer de niveau social. Tu as execercé comme avocat, enseignant à la faculté, et maintenant tu es le conseiller et le précepteur du jeune Alto Richèse.
Tu peux ainsi loger ta femme et tes deux filles aux palais du Comte. Tes faits d'armes sont encore présent dans l'esprit des plus jeune de la planète. Ils ont tous une respect certains pour toi. Cela commence à être loin pour toi tout cela. Tu n'es plus dans la forme physique de l'époque, ton esprit tacticien est lui toujours aussi vif.   
}
\end{description}
		\begin{itemize}
		\item[Alto Richèse]: C'est un jeune homme que tu qualifies de "distrait", il semble avoir un grand intérêt pour ce que tu lui enseignes, il y montre une certaine compétence. Il pourrait faire un bon seigneur. Il est déjà un fin politicien malgrés son ages et son désintérêt de tout cela. Il rêve de vivra sans responsabilité.  
		\item[Edward Linoilis: le maitre d'armes]: C'est un frimeur, qui passe sont temps à séduire les jeunes filles de peu de vertues. Il est cependant très habile avec un bouclier et une lame. Il aimerait bien t'affronter car pour lui, tu es une légende. Il ne fait pas l'ombre d'un doute qui te battra sans un effort. C'est une personne assez simple, et surement très loyale si on y met le bon prix.  
		\item[Octave Siav: Professeur de pilotage ]: C'est une personne très mystérieuse. Il cache clairement son jeu. C'est un pilote moyen, mais il doit être dans les bons papiers du comte ou un truc comme ça. Il n'est pas bavard mais fait son travail avec enthousiasme, en dépis de la médiocrité des compétences de pilotage du jeune Alto.
		\end{itemize} 
\clearpage
\clearpage

\section{Espion}
\begin{description}
\item[Nom:]{Octave Siav}
\item[Origine:]{KarKall}
\item[Profession:]{Espion}
\item[Age:]{28 ans}
\item[Histoire:]{
tu es l'espion qui monte aux yeux de la famille Teozys, ta couverture est d'être son professeur de pilotage, ton vrai rôle est d'apprendre tout sur lui et ses contacts. Il est évident que tu dois aussi être son garde du corps. Tu communiques le plus possible avec le Comte Teozys. Tu es la sécurité supplémentaire, le fusible de secours.
	Tu t'ennuies un peu dans cette mission, tes talents pourrait être mieux utiliser ailleurs. Tu ne comprends pas trop l'intérêt que represente la protection et l'éducation du jeune Alto. 
Dans tous les cas, tu preferes effectuer ta mission dans les règles de l'art, peut-être que la suivante sera plus intéressante.
}
\end{description}
		\begin{itemize}
		\item[Alto Richèse]: jeune garçon un peu distrait, tu n'arrives pas à lui enseigner les règles du pilotage. Ce n'est pas une personne d'action, tu es gagné  sa confiance. Il se confie un peu à toi.  
		\item[Edward Linoilis: le maitre d'armes]: Jeune frimeur, il n'est vraiment pas un maitre de discretion. Le roi de la débauche, il maitrise vraiment son art, tu n'aimerais pas te retrouver face à lui dans une arène. 
		\item[Penwyr Lanton : le prof particulier]:  Son passé et son esprit sont les pires ennemies de ta couverture. Il a souvent fait des réflexions pour te faire comprendre qu'il te soupçonnait de quelques choses. Tu restes prudent avec lui. Il est bien trop important pour le comte Teozys, tu ne peux pas te débarrasser de lui.
		\item[Aly'o Erigann : Esclave ]: Esclave et servante du jeune Alto, elle est plus intelligente qu'elle ne laisse paraître son niveau social.     

		\end{itemize} 



\clearpage
\section{Courtisanne}
\begin{description}
\item[Nom:]{Alécia Veyron}
\item[Origine:]{Karkall}
\item[Profession:]{Courtissanne}
\item[Age:]{16 ans}
\item[Histoire:]{
Tu es né la même année que le prince Alto, ton père (Phillipus) est l'intendant de la famille, Il fonde en toi beaucoup d'espoir. 
Il espère que tu sera la première de la famille à devenir une noble. Il espère que tu épouseras le jeune prince. Ainsi la famille Veyron s'attachera avec la richissime famille Richèse. 
Le bon travail de ton père au près de Salmus Richèse (le duc au pouvoir) était le principal argument pour que le duc accepte cet union. 
Pour des raisons de sécurité et d'autres que tu ne comprends pas, le Duc a envoyé son fils en exil sur la planète: Kankall, il y a 2 ans, maintenant. 
Ton père t'a chargé d'aller à la rencontre d'Alto pour lui apprendre la nouvelle de la mort de son père. 
Son oncle a pris le pouvoir et du coup, le travail de ton père ainsi que tous ses espoirs lui sont tombés à l'eau. 
Sa dernière chance est de revoir Alto venir et récupérer son trône. 
Ton père et le duc étaient les seuls à connaitre l'emplacement d'Alto. Il t'envoie ainsi apporter le message. 
}
\end{description}
		\begin{itemize}
		\item[Alto Richèse]: Il est beauuuuuuu! depuis ton plus jeune age ton père t'a éduqué pour qu'un jour tu deviennes sa femme. Tu as accepté ce destin et tu es prête à tout pour que cela devienne réalité. Cependant tu n'es pas la seule en lisse.  
		\item[Aly'o Erigann]: Esclave d'Alto: Il lui attache trop d'importance pour une esclave. 
	\end{itemize} 




\chapter{L'histoire commence}
\section{Convocation}
Le  jeune héritier est avec sa cour, dans ses appartements. Un valet arrive et déclare que le jeune seigneur est attendu au plus vite. 
Les joueurs sont conviés à venir avec lui. 

\section{Mauvaise nouvelle}
L'héritier apprend alors que son père est mort. Le décès est naturel mais une enquête est en cours. C'est Alécia qui est venue en personne apporter la nouvelle. 
Il est donc rappelé dans son domaine, son oncle assure le pouvoir épaulé par la mère de l'héritier.
Il prend vraiment mal l'information et se réfugie dans ses appartements, en hurlant de vouloir rester seul. 
Cela laisse le temps au comte Teozus de parler avec les PJ. 

Il est possible que seule l'esclave soit capable de raisonner le prince.

\section{Départ}
Possible RP avec un navigateur de la guilde. 

\section{Arrivée}
Oncle Margus accueille Alto par une cérémonie assez militaire. Il a effectivement pris le pouvoir et se montre en public avec sa belle-soeur. 
L'oncle fait un discours traitant l'enfant de trop jeune pour régner et répète que d'ici quelques années, il reprendra le pouvoir.

\section{Attentat}
Puis les invités sont conviés à un diner. Sur le trajet vers la salle de banquet, un attentat est perpétré contre le jeune héritier. 
Si les PJ n'interviennent pas,  "de mystérieux guerrière se sacrifiront à la place".
Un homme avec un poignard, il avalera une gellule empoisonée pour ne pas parler. 

\section{enquête}
A partir de là, Alto doit évoluer seul, il demandera à ses amis de mener une enquête sur ses assassins et sur son père. Il lui faut une protection rapprocher et un plan.
Il demandera sûrement appuie de la famille Veyron. 

\section{Guérison}
La mêre d'Alto viendra le voir en presence des PJ (peut-être pas tous).

\section{Le père a bien été assassiné}
Les joueurs pourront apprendre que les medecins ont bien confirmé l'assassinat de Salmus. Une trace d'aiguille a été trouvé dans la jambe gauche.
Le peuple et la cour désigne obligatoirement la famille Vilnus (Ixien). Une plainte est déposé au près du Landstraad. 

\section{L'oncle demande conseil}
L'oncle vient dans les appartements du jeune héritier pour demander conseil afin de pousser le Landstraad à punir la famille Vilnus
